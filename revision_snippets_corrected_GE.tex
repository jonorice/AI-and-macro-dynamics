
% ======================================================================
% REVISION SNIPPETS (corrected GE linearisation)
% Copy and paste the blocks below into your main .tex.
% ======================================================================

% ----------------------------------------------------------------------
% A. FIGURE NOTES (Figures 1–3)
% ----------------------------------------------------------------------
% Replace any sentence claiming the GE and PE paths are "visually indistinguishable".
% With the corrected GE linearisation, the two paths generally diverge (especially on impact),
% so the benchmark lines should be visible without special plotting tricks.
%
% Suggested note text:
%
% Figure 1 notes: "GE is the local stable-manifold solution. PE keeps replacement investment fixed."
% Figure 2 notes: "GE re-evaluates the local transition at the new steady state after the longevity shock."
% Figure 3 notes: "GE is a nonlinear perfect-foresight transition anchored at the steady state; PE is the
%                  replacement-investment diffusion benchmark."

% ----------------------------------------------------------------------
% B. APPENDIX B REPLACEMENT
% ----------------------------------------------------------------------
% Replace your ENTIRE current Appendix B (from "\section{Jacobian, eigenvalues and local dynamics}"
% up to the end of Appendix B) with the block below.

\section{Jacobian, eigenvalues and local dynamics}
\label{app:jacobian}

This appendix summarises the construction of the local general-equilibrium dynamics used in
Section~\ref{sec:dynamics}. Away from the steady state, gross investment is not equal to depreciation,
so the linearisation must keep investment explicit in the feasibility constraint. Once this is done,
the Euler block is second order in the predetermined stocks $(K_t,M_t)$ because $\lambda_t$ depends on
$t$-dated consumption, and consumption depends on the chosen next-period capital stocks.

\medskip

\noindent Throughout, we focus on the empirically relevant binding-capacity regime in which
$D_t=\psi M_t^{\phi}$ and $\mathrm{MRS}_t>\chi$ at the steady state.

% Number equations as B.1, B.2, ...
\setcounter{equation}{0}
\renewcommand{\theequation}{B.\arabic{equation}}

\subsection{Equilibrium conditions used for the linearisation}

Under binding capacity, the relevant equilibrium conditions are
\begin{align}
C^{\mathrm T}_t + I^K_t + X_t + \chi D_t &= Y_t, \label{eq:B_resource}\\
K_{t+1} &= (1-\delta)K_t + I^K_t, \label{eq:B_Klaw}\\
M_{t+1} &= (1-\delta_M)M_t + X_t, \label{eq:B_Mlaw}\\
Y_t &= A K_t^{\alpha} M_t^{\zeta}, \qquad D_t=\psi M_t^{\phi}, \label{eq:B_prodcap}\\
C_t &= \Big[\theta (C^{\mathrm T}_t)^{\rho} + (1-\theta) D_t^{\rho}\Big]^{1/\rho},
\qquad \rho\equiv\frac{\sigma-1}{\sigma}, \label{eq:B_CES}\\
\mathrm{MRS}_t &= \frac{1-\theta}{\theta}\Big(\frac{C^{\mathrm T}_t}{D_t}\Big)^{1/\sigma}, \label{eq:B_MRS}
\end{align}
together with the Euler equations
\begin{align}
\lambda_t &= \beta\,\lambda_{t+1}\Big[(1-\delta)+\alpha\,\frac{Y_{t+1}}{K_{t+1}}\Big], \label{eq:B_eulerK}\\
\lambda_t &= \beta\,\lambda_{t+1}\Big[(1-\delta_M)+\zeta\,\frac{Y_{t+1}}{M_{t+1}}
+(\mathrm{MRS}_{t+1}-\chi)\,\phi\psi\,M_{t+1}^{\phi-1}\Big], \label{eq:B_eulerM}
\end{align}
where $\lambda_t$ is the shadow value of one unit of tangible goods in utility units,
$\lambda_t = U_C(C_t)\,\partial C_t/\partial C^{\mathrm T}_t$.

\paragraph{Why replacement investment cannot be used away from steady state}
In the steady state, $I^{K*}=\delta K^{*}$ and $X^{*}=\delta_M M^{*}$, but these identities do not hold
along transitions. Substituting them into the non-steady-state feasibility condition would eliminate the
channel through which investment moves $t$-dated consumption and therefore the co-state $\lambda_t$.
Equations \eqref{eq:B_resource}--\eqref{eq:B_eulerM} retain the correct dependence of $\lambda_t$ on
investment, which is essential for the local dynamics.

\subsection{Second-order Euler block and reduced state transition}

Let $k_t\equiv\ln K_t-\ln K^{*}$ and $m_t\equiv\ln M_t-\ln M^{*}$ denote log deviations from the
steady state, and collect them in $x_t\equiv(k_t,m_t)^{\prime}$. Because $\lambda_t$ depends on
$(K_t,M_t,K_{t+1},M_{t+1})$ via \eqref{eq:B_resource}--\eqref{eq:B_CES}, log-linearising the Euler block
yields a second-order linear difference equation in $x_t$,
\begin{equation}
A\,x_{t+2} + B\,x_{t+1} + C\,x_t = 0,
\label{eq:B_second_order}
\end{equation}
where $A$, $B$ and $C$ are $2\times2$ matrices of steady-state elasticities.

A convenient way to obtain the reduced two-state law of motion is to form the $4\times4$ companion system
for $(x_{t+1},x_t)$ and select the stable eigenspace. Saddle-path stability implies two eigenvalues inside
the unit circle, matching the number of predetermined stocks. The resulting local stable-manifold
transition takes the form
\begin{equation}
x_{t+1} = J\,x_t.
\label{eq:B_GE_reduced}
\end{equation}

\subsection{Baseline numerical Jacobian and eigenvalues}

Under the baseline calibration in Table~\ref{tab:baseline_calib}, the computed reduced-form Jacobian is
\begin{equation}
J \approx
\begin{pmatrix}
0.947 & 0.061\\
0.507 & 0.033
\end{pmatrix},
\label{eq:B_J_numeric}
\end{equation}
with eigenvalues
\begin{equation}
\lambda_1 \approx 7.7\times10^{-6},\qquad \lambda_2 \approx 0.980.
\label{eq:B_eigs_numeric}
\end{equation}
The near-zero stable root implies a very rapid adjustment mode that operates through endogenous
investment choices for the next-period stocks, while the second root governs the slower component of the
local convergence. The updated Figures~1--3 report the implied GE paths alongside the depreciation-based
partial-equilibrium benchmarks.
